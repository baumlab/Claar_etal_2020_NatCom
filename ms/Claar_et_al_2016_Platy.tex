\documentclass[]{article}
\usepackage{lmodern}
\usepackage{amssymb,amsmath}
\usepackage{ifxetex,ifluatex}
\usepackage{fixltx2e} % provides \textsubscript
\ifnum 0\ifxetex 1\fi\ifluatex 1\fi=0 % if pdftex
  \usepackage[T1]{fontenc}
  \usepackage[utf8]{inputenc}
\else % if luatex or xelatex
  \ifxetex
    \usepackage{mathspec}
  \else
    \usepackage{fontspec}
  \fi
  \defaultfontfeatures{Ligatures=TeX,Scale=MatchLowercase}
\fi
% use upquote if available, for straight quotes in verbatim environments
\IfFileExists{upquote.sty}{\usepackage{upquote}}{}
% use microtype if available
\IfFileExists{microtype.sty}{%
\usepackage{microtype}
\UseMicrotypeSet[protrusion]{basicmath} % disable protrusion for tt fonts
}{}
\usepackage[margin=1in]{geometry}
\usepackage{hyperref}
\hypersetup{unicode=true,
            pdftitle={KI Platygyra Manuscript},
            pdfauthor={Danielle C Claar, Kristina L Tietjen, Ruth D Gates, Julia K Baum},
            pdfborder={0 0 0},
            breaklinks=true}
\urlstyle{same}  % don't use monospace font for urls
\usepackage{graphicx,grffile}
\makeatletter
\def\maxwidth{\ifdim\Gin@nat@width>\linewidth\linewidth\else\Gin@nat@width\fi}
\def\maxheight{\ifdim\Gin@nat@height>\textheight\textheight\else\Gin@nat@height\fi}
\makeatother
% Scale images if necessary, so that they will not overflow the page
% margins by default, and it is still possible to overwrite the defaults
% using explicit options in \includegraphics[width, height, ...]{}
\setkeys{Gin}{width=\maxwidth,height=\maxheight,keepaspectratio}
\IfFileExists{parskip.sty}{%
\usepackage{parskip}
}{% else
\setlength{\parindent}{0pt}
\setlength{\parskip}{6pt plus 2pt minus 1pt}
}
\setlength{\emergencystretch}{3em}  % prevent overfull lines
\providecommand{\tightlist}{%
  \setlength{\itemsep}{0pt}\setlength{\parskip}{0pt}}
\setcounter{secnumdepth}{5}
% Redefines (sub)paragraphs to behave more like sections
\ifx\paragraph\undefined\else
\let\oldparagraph\paragraph
\renewcommand{\paragraph}[1]{\oldparagraph{#1}\mbox{}}
\fi
\ifx\subparagraph\undefined\else
\let\oldsubparagraph\subparagraph
\renewcommand{\subparagraph}[1]{\oldsubparagraph{#1}\mbox{}}
\fi

%%% Use protect on footnotes to avoid problems with footnotes in titles
\let\rmarkdownfootnote\footnote%
\def\footnote{\protect\rmarkdownfootnote}

%%% Change title format to be more compact
\usepackage{titling}

% Create subtitle command for use in maketitle
\newcommand{\subtitle}[1]{
  \posttitle{
    \begin{center}\large#1\end{center}
    }
}

\setlength{\droptitle}{-2em}
  \title{KI Platygyra Manuscript}
  \pretitle{\vspace{\droptitle}\centering\huge}
  \posttitle{\par}
  \author{Danielle C Claar, Kristina L Tietjen, Ruth D Gates, Julia K Baum}
  \preauthor{\centering\large\emph}
  \postauthor{\par}
  \predate{\centering\large\emph}
  \postdate{\par}
  \date{17 September, 2016}

\begin{document}
\maketitle

\section{Summary}\label{summary}

\emph{Summary goes here. Why are the Platys not bleached anymore}
\emph{150 words - Up to 150 words, no references,numbers, abbreviations,
acronyms or measurements unless essential. It contains 2--3 sentences of
basic-level introduction to the field; a brief account of the background
and rationale of the work; a statement of the main conclusions
(introduced by the phrase `Here we show' or its equivalent); and 2--3
sentences putting the main findings into general context so it is clear
how the results described have moved the field forwards} Basic level
introduction 1 (corals and climate change) Basic level introduction 2
(what about the future of general resilience to climate change) Basic
level introduction 3 (How might corals be resilient - Symbiodinium)
Background 1 (coral bleaching debate) Background 2 (corals switching
symbionts is good or bad?) Background 3 (corals can regain symbionts
ONLY AFTER stress subsides) Here we show (different dominant symbionts
drive symbiont diversity capacity) and (those pesky \textless{}2\%
symbionts are indeed important) and (corals can regain symbionts during
an extended stress event - hopeful!) Now fit it into a general context

\section{\texorpdfstring{Background \emph{Need to rename using header
requirements: The text may contain subheadings (less than six in total)
of less than 40 characters (including spaces)
each.}}{Background Need to rename using header requirements: The text may contain subheadings (less than six in total) of less than 40 characters (including spaces) each.}}\label{background-need-to-rename-using-header-requirements-the-text-may-contain-subheadings-less-than-six-in-total-of-less-than-40-characters-including-spaces-each.}

Intro goes here. Kiritimati, El Nino, coral bleaching, Platygyra,
recovery, resilience \emph{Articles are typically 3,000 words of text
(not including Methods, summary or other sections), beginning with up to
500 words of referenced text expanding on the background to the work,
before proceeding to a concise, focused account of the findings, ending
with one or two short paragraphs of discussion.} (corals and climate
change) (what about the future of general resilience to climate change)
(El Ninos are terrible, this one was the worst so far, and they are
getting worse) (How might corals be resilient - Symbiodinium) (coral
bleaching debate) (corals switching symbionts is good or bad?) (corals
can regain symbionts ONLY AFTER stress subsides which limits coral
recovery when stress increases) Here, we tagged and sampled the same
corals before, during, and immediately after the el nino event, on
Kiritimati (something about Kiritimati). We used Illumina sequencing to
evaluate changes in symbiodinium community structure coincident with the
2015-2016 major el nino event. The goal was to understand\ldots{}why the
hell these corals survived 10 months of extreme heat stress, and
actually got better in the middle of it.

\section{Findings}\label{findings}

\emph{a concise, focused account of the findings, probably
\textless{}2,000 words} Results Paragraph 1 (different dominant
symbionts drive symbiont diversity capacity, also they didn't switch at
commencement of bleaching event) \emph{Figure about diversity} Results
Paragraph 2 (\textless{}2\% symbionts are indeed important) - can I
calculate some sort of change score for the different types?
\emph{Figure of example sequence abundances, superimposed on coral 99
images} Results Paragraph 3 (Corals can regain symbionts during an
extended stress event)

\section{Discussion}\label{discussion}

\emph{one or two short paragraphs of discussion. Probably around 500
wds} Discussion Paragraph 1 (Methods like NGS are really important for
understanding these changes in symbiont diversity, as well as for seeing
those low aboundance symbionts) Discussion Paragraph 2 (What does their
recovery tell us about the future of coral reefs?)\ldots{} Why are
Platys so excellent and what does that tell us about when coral
resilience is threatened by extreme climatic events?

\section{Methods}\label{methods}

\emph{The Methods section should be written as concisely as possible but
should contain all elements necessary to allow interpretation and
replication of the results. As a guideline, Methods sections typically
do not exceed 3,000 words.}

\subsection{Field Information}\label{field-information}

Kiritimati basics - located in the Central Equatorial Pacific, smack dab
in the middle of the Nino 3.4 region (used to quantify el nino presence
and strength), human disturbance gradient, bleaching event there (cite
bleaching paper here) Tagging corals and collecting samples - transects,
tagging corals, photoing corals, sampling corals, processing samples,
storage in Guanidinium Taxa sampled - platy, favites, favia, etc. \#\#
Pre-processing and sequencing DNA Extraction - extraction protocol ITS2
region - it's annoying, but it's the best we've got right now PCR and
Cleanup - Amy's method of PCR and cleanup Library Prep - Amy's method of
library prep, include Illumina Sequencing information (barcodes, etc)
\#\# Post Processing Sequence QC - boku, then merge with illumina utils,
max mismatch=3 Sequence clustering - denovo clustering using UCLUST in
QIIME, then compare to reference database to assign taxonomy Statistical
Analysis - alpha diversity of sequence reads, co-occurence?, beta
diversity?

\section{References}\label{references}


\end{document}
