\documentclass[]{article}
\usepackage{lmodern}
\usepackage{amssymb,amsmath}
\usepackage{ifxetex,ifluatex}
\usepackage{fixltx2e} % provides \textsubscript
\ifnum 0\ifxetex 1\fi\ifluatex 1\fi=0 % if pdftex
  \usepackage[T1]{fontenc}
  \usepackage[utf8]{inputenc}
\else % if luatex or xelatex
  \ifxetex
    \usepackage{mathspec}
  \else
    \usepackage{fontspec}
  \fi
  \defaultfontfeatures{Ligatures=TeX,Scale=MatchLowercase}
\fi
% use upquote if available, for straight quotes in verbatim environments
\IfFileExists{upquote.sty}{\usepackage{upquote}}{}
% use microtype if available
\IfFileExists{microtype.sty}{%
\usepackage{microtype}
\UseMicrotypeSet[protrusion]{basicmath} % disable protrusion for tt fonts
}{}
\usepackage[margin=1in]{geometry}
\usepackage{hyperref}
\hypersetup{unicode=true,
            pdftitle={KI Platygyra Manuscript},
            pdfauthor={Danielle C Claar, Kristina L Tietjen, Ruth D Gates, Julia K Baum},
            pdfborder={0 0 0},
            breaklinks=true}
\urlstyle{same}  % don't use monospace font for urls
\usepackage{graphicx,grffile}
\makeatletter
\def\maxwidth{\ifdim\Gin@nat@width>\linewidth\linewidth\else\Gin@nat@width\fi}
\def\maxheight{\ifdim\Gin@nat@height>\textheight\textheight\else\Gin@nat@height\fi}
\makeatother
% Scale images if necessary, so that they will not overflow the page
% margins by default, and it is still possible to overwrite the defaults
% using explicit options in \includegraphics[width, height, ...]{}
\setkeys{Gin}{width=\maxwidth,height=\maxheight,keepaspectratio}
\IfFileExists{parskip.sty}{%
\usepackage{parskip}
}{% else
\setlength{\parindent}{0pt}
\setlength{\parskip}{6pt plus 2pt minus 1pt}
}
\setlength{\emergencystretch}{3em}  % prevent overfull lines
\providecommand{\tightlist}{%
  \setlength{\itemsep}{0pt}\setlength{\parskip}{0pt}}
\setcounter{secnumdepth}{5}
% Redefines (sub)paragraphs to behave more like sections
\ifx\paragraph\undefined\else
\let\oldparagraph\paragraph
\renewcommand{\paragraph}[1]{\oldparagraph{#1}\mbox{}}
\fi
\ifx\subparagraph\undefined\else
\let\oldsubparagraph\subparagraph
\renewcommand{\subparagraph}[1]{\oldsubparagraph{#1}\mbox{}}
\fi

%%% Use protect on footnotes to avoid problems with footnotes in titles
\let\rmarkdownfootnote\footnote%
\def\footnote{\protect\rmarkdownfootnote}

%%% Change title format to be more compact
\usepackage{titling}

% Create subtitle command for use in maketitle
\newcommand{\subtitle}[1]{
  \posttitle{
    \begin{center}\large#1\end{center}
    }
}

\setlength{\droptitle}{-2em}
  \title{KI Platygyra Manuscript}
  \pretitle{\vspace{\droptitle}\centering\huge}
  \posttitle{\par}
  \author{Danielle C Claar, Kristina L Tietjen, Ruth D Gates, Julia K Baum}
  \preauthor{\centering\large\emph}
  \postauthor{\par}
  \predate{\centering\large\emph}
  \postdate{\par}
  \date{20 September, 2016}

\begin{document}
\maketitle

\begin{verbatim}
## Loading required package: qdapDictionaries
\end{verbatim}

\begin{verbatim}
## Loading required package: qdapRegex
\end{verbatim}

\begin{verbatim}
## Loading required package: qdapTools
\end{verbatim}

\begin{verbatim}
## Loading required package: RColorBrewer
\end{verbatim}

\begin{verbatim}
## 
## Attaching package: 'qdap'
\end{verbatim}

\begin{verbatim}
## The following object is masked from 'package:base':
## 
##     Filter
\end{verbatim}

\begin{verbatim}
## [1] NA
\end{verbatim}

\section{Summary}\label{summary}

Ocean ecosystems worldwide are threatened by climate change-induced
increases in seawater temperatures. Pulse warming events such as El Nino
amplify these threats, causing massive losses of coral cover (e.g.~17\%
of coral reefs during the 1997/98 El Nino). The 2015/16 El Nino is
currently the worst on record in terms of severity and longevity, yet
despite massive coral mortality, some corals show resilience to this
extreme event. Coral resilience is related to many factors, including
the structure and flexibility of their internal symbiotic communities.
Here we show that the ability of a coral to house a diverse suite of
symbionts (\emph{Symbiodinium}) is driven by the identity of the
dominant \emph{Symbiodinium} type. Additionally we show that, contrary
to current opinion, symbionts present in miniscule abundances
(\textless{}2\%) are indeed important for coral recovery, and
furthermore, that corals have the ability to regain symbionts during an
extended stress event, providing hope for the future of coral
resilience.

\section{\texorpdfstring{Background \emph{rename using header reqs: The
text may contain subheadings (\textless{}6 in total) of \textless{}40
characters (incl spaces)
each.}}{Background rename using header reqs: The text may contain subheadings (\textless{}6 in total) of \textless{}40 characters (incl spaces) each.}}\label{background-rename-using-header-reqs-the-text-may-contain-subheadings-6-in-total-of-40-characters-incl-spaces-each.}

El Nino warming is terrible, creating a particularly acute threat to
coral reefs, which already live on the edge of their thermal tolerance.
El Niño, the positive phase of the ENSO (El Niño Southern Oscillation),
is a natural climatic event that occurs when surface waters heat up in
the equatorial Pacific, causing catastrophic effects on reef ecosystems
by disrupting coral symbioses (i.e.~coral bleaching). The 2015-2016 El
Niño is the first major global event since 1997-1998, and has been
declared the third global coral bleaching event by NOAA (cite Reef
article). El Ninos are getting worse, threatening coral reefs and
endangering the persistence of vital ecosystem services, threatening the
food security and coastline protection of coastal communities worldwide.
Kiritimati Atoll (Christmas Island), located in the Central Equatorial
Pacific, is at the epicenter of this extreme El Niño event. Thermal
anomalies were severe on Kiritimati, reaching an unprecedented (cite o
h-g) XX number of DHW over a XX month long bleaching event, demolishing
most of the reef (\textbf{???}). \emph{as stressors increase worldwide,
all reefs (from shit to pristine) will be increasingly threatened as
resilience and recovery are limited by overwhelming environmental
stress} Kiritimati reefs range from heavily impacted to nearly pristine,
and thus provides an ideal microcosm to investigate how reefs worldwide
will respond to increasing environmental stress, and \emph{XX} potential
capacity for resilience and recovery.

El nino warming threatens coral reefs by disrupting the dynamic
symbiosis between coral and their internal symbiotic algae
(Symbiodinium). This symbiosis is the foundation of reef ecosystems, and
a critical element of reef resilience (van oppen and gates 2006). Corals
host a diverse community of Symbiodinium, ranging along a continuum from
`selfish opportunistic symbionts' (e.g.~some clade D Symbiodinium) which
are better suited to sustained environmental stress than others, to
`intimately evolved symbionts' which provide exceptional amounts of
nutrition to their coral host (1). Thus, although these relationships
have developed over evolutionary time, the resilience of the coral
symbiome is constantly shaped by dynamic coral-symbiont interactions
(2). \emph{We used to think that bleaching might be good - ABH says that
corals bleach in order to expel suboptimal Symbiodinium types in
exchange for optimal symbionts during the new conditions (3,
Baker:2001bf, 4)} \emph{We do know that corals house background
symbionts in low abundances (5, all the recent ngs studies\ldots{}), but
these relationships have been described as unstable (6, more cites?).
}Switching and shuffling (7)* \emph{And we do know that some symbio are
``better'' than others} \emph{And then we said that bleaching is
definitely bad} \emph{But at least we do know that it allows changes to
occur in the Symbiodinium community structure}

Example of when this high-risk ecological opportunity (8) actually pays
off\ldots{}

\emph{Regardless it is likely that symbiodinium community composition is
important for resilience, corals that host flexible symbioses may be
more sensitive to environmental changes (9)} \emph{Taxa-specific
bleaching is a thing} \emph{When we've seen resilience/recovery before,
corals have only been demonstrated to recover if the stress goes away
first} Previously, corals have been shown to recover from bleaching only
after the external stress (e.g.~warming) has subsided. \emph{implying
that longer and more frequent stressors spell disaster for reefs
worldwide}

Here, we tagged and sampled the same corals before, during, and
immediately after the el nino event, on Kiritimati (something about
Kiritimati). We used Illumina sequencing to evaluate changes in
symbiodinium community structure coincident with the 2015-2016 major el
nino event. The goal was to understand\ldots{}why the hell these corals
survived 10 months of extreme heat stress, and actually got better in
the middle of it.

\section{Findings}\label{findings}

\emph{a concise, focused account of the findings, probably
\textless{}2,000 words} Results Paragraph 1 (different dominant
symbionts drive symbiont diversity capacity, also they didn't switch at
commencement of bleaching event) \emph{Figure about diversity} Results
Paragraph 2 (\textless{}2\% symbionts are indeed important) - can I
calculate some sort of change score for the different types?
\emph{Figure of example sequence abundances, superimposed on coral 99
images} Results Paragraph 3 (Corals can regain symbionts during an
extended stress event)

\section{Discussion}\label{discussion}

\emph{one or two short paragraphs of discussion. Probably around 500
wds} Discussion Paragraph 1 (Methods like NGS are really important for
understanding these changes in symbiont diversity, as well as for seeing
those low aboundance symbionts) Discussion Paragraph 2 (What does their
recovery tell us about the future of coral reefs?)\ldots{} Why are
Platys so excellent and what does that tell us about when coral
resilience is threatened by extreme climatic events?

Elucidating the mechanisms underlying changes in coral-symbiont
interactions is essential to understanding the ability of the coral
symbiome to adapt to the multiple stressors they now face.

\section{Methods}\label{methods}

\emph{The Methods section should be written as concisely as possible but
should contain all elements necessary to allow interpretation and
replication of the results. As a guideline, Methods sections typically
do not exceed 3,000 words.}

\subsection{Field Information}\label{field-information}

Kiritimati basics - located in the Central Equatorial Pacific, smack dab
in the middle of the Nino 3.4 region (used to quantify el nino presence
and strength), human disturbance gradient, bleaching event there (cite
bleaching paper here) Tagging corals and collecting samples - transects,
tagging corals, photoing corals, sampling corals, processing samples,
storage in Guanidinium Taxa sampled - platy, favites, favia, etc. \#\#
Pre-processing and sequencing DNA Extraction - extraction protocol ITS2
region - it's annoying, but it's the best we've got right now PCR and
Cleanup - Amy's method of PCR and cleanup Library Prep - Amy's method of
library prep, include Illumina Sequencing information (barcodes, etc)
\#\# Post Processing Sequence QC - boku, then merge with illumina utils,
max mismatch=3 Sequence clustering - denovo clustering using UCLUST in
QIIME, then compare to reference database to assign taxonomy Statistical
Analysis - alpha diversity of sequence reads, co-occurence?, beta
diversity?

\section{References}\label{references}

\hypertarget{refs}{}
\hypertarget{ref-Lesser:2013tu}{}
1. Lesser, M. P., Stat, M. \& Gates, R. D. The endosymbiotic
dinoflagellates (\emph{Symbiodinium} sp.) of corals are parasites and
mutualists. \emph{Coral Reefs} 1--9 (2013).

\hypertarget{ref-Stat:2006ww}{}
2. Stat, M., Carter, D. \& Hoegh-Guldberg, O. The evolutionary history
of \emph{Symbiodinium} and scleractinian hosts --- symbiosis, diversity,
and the effect of climate change. \emph{Perspectives in Plant Ecology}
\textbf{8,} 23--43 (2006).

\hypertarget{ref-Buddemeier:1993bb}{}
3. Buddemeier, R. W. \& Fautin, D. G. Coral bleaching as an adaptive
mechanism. \emph{Bioscience} \textbf{43,} 320--326 (1993).

\hypertarget{ref-Buddemeier:2004vj}{}
4. Buddemeier, R. W., Baker, A. C., Fautin, D. G. \& Jacobs, J. R. in
(Coral health; \ldots{}, 2004).

\hypertarget{ref-Correa:2009jy}{}
5. Correa, A. M. S., McDonald, M. D. \& Baker, A. C. Development of
clade-specific \emph{Symbiodinium} primers for quantitative PCR (qPCR)
and their application to detecting clade D symbionts in Caribbean
corals. \emph{Marine biology} \textbf{156,} 2403--2411 (2009).

\hypertarget{ref-Coffroth:2010ju}{}
6. Coffroth, M. A., Poland, D. M., Petrou, E. L., Brazeau, D. A. \&
Holmberg, J. C. Environmental symbiont acquisition may not be the
solution to warming seas for reef-building corals. \emph{PLoS ONE}
\textbf{5,} e13258 (2010).

\hypertarget{ref-Baker:2003uo}{}
7. Baker, A. C. Flexibility and specificity in coral-algal symbiosis:
diversity, ecology, and biogeography of Symbiodinium. \emph{Annual
Review of Ecology, Evolution, and Systematics} 661--689 (2003).

\hypertarget{ref-Baker:2001bf}{}
8. Baker, A. C. Ecosystems: Reef corals bleach to survive change.
\emph{Nature} \textbf{411,} 765--766 (2001).

\hypertarget{ref-Putnam:2012bn}{}
9. Putnam, H. M., Stat, M., Pochon, X. \& Gates, R. D. Endosymbiotic
flexibility associates with environmental sensitivity in scleractinian
corals. \emph{Proceedings of the Royal Society B: Biological Sciences}
\textbf{279,} 4352--4361 (2012).


\end{document}
